\chapter{Úvod}
\label{úvod}
	Za posledních několik let roste popularita genealogie směrem nahoru. Lidé chtějí znát kam až vedou kořeny jejich rodokmenu, zjistit různé informace o~svých předcích a nebo genealogii považují za svůj koníček. Ovšem postup vyhledávání takových informací není vůbec jednoduchou záležitostí. Aby badatel dohledal všechny potřebné informace musí si projít dlouhým procesem vybírání vhodných matrik, ve kterých by dané informace získal. Musí určit všechny oblasti, ve kterých se osoba vyskytovala, časové rozmezí a nahlížet při tom i na jeho příbuzné. 
	
	Tato práce se snaží zjednodušit tento zdlouhavý postup. Jejím cílem je navrhnout a zrealizovat webovou aplikaci, která by uživatelům zjednodušila proces vyhledávání chybějících informací o~konkrétních osobách. Aplikace bude jako vstup zpracovávat soubor o~formátu GEDCOM, vytvoří záznamy o~chybějících informacích k~daným osobám a k~těmto záznamům navrhne vhodné matriky na základě oblastí a časových rozmezí zjištěných ze známých informací dané osoby, či jejich příbuzných. Zvolené matriky pak budou uživateli nabídnuty prostřednictvím uživatelského prostředí a bude možné se přes ně navigovat na samotnou digitalizovanou podobu matriční knihy. Uživatel bude mít dále možnost zapisovat si k~záznamům poznámky, nebo záznamy ignorovat. 
	
	Práce je členěna do 4 částí. První část dokumentu se zaobírá teoretickými znalostmi, potřebnými ke správnému návrhu a implementaci aplikace. Je zde vysvětlen pojem genealogie a s~ním pojmy svázané. Jsou zde popsány matriční knihy a jejich digitální vyobrazení. A~jako poslední je zde stručný popis GEDCOM formátu, jenž působí jako vstup pro naši aplikaci.
	
	V~druhé části dokumentu jsou popsány technologie použité při implementaci aplikace, včetně důležitých knihoven, modulů, či frameworků.
	
	Třetí část se věnuje analýze a návrhu aplikace. Tato část opět apeluje na cíle práce. Dále provádí analýzu uživatelů, pro vyobrazení funkčností, kterých by měla aplikace nabývat. Důležitý je také popis návrhu databázové struktury vhodné pro ukládání genealogických dat, rovněž obsažený v~této části. Závěr kapitoly je věnován jednoduchému návrhu samotného grafického uživatelského rozhraní. 
	
	Čtvrtá část obsahuje popis jednotlivých fází implementace aplikace. Jsou zde popsané fáze od přípravy potřebných dat, zpracování jednotlivých skriptů až po implementaci samotné aplikace.