\chapter{Závěr}

Cílem této práce bylo vytvořit aplikaci, která umožní uživateli nahrát soubor ve formátu GEDCOM, zobrazí mu vytvořené záznamy k~chybějícím informacím událostí narození, úmrtí i oddání a navrhne matriky kde by tyto informace mohl nalézt. 

V~rámci jejího vypracování bylo zapotřebí nastudovat problematiku provádění genealogického výzkumu, strukturu matrik a formátu GEDCOM. Bylo potřeba analyzovat požadavky uživatelů, navrhnout vzhled aplikace, strukturu databáze a zvolit vhodné technologie k~vypracování aplikace. Pomocí těchto technologií pak implementovat její jednotlivé části.

Součástí této práce nebyla jen implementace aplikace, ale také příprava a zpracování všech potřebných dat. Tedy pochopení a zpracování územního rozsahu České republiky, a dále extrakce dat ze všech osmi archivů, uchovávajících matriční knihy, do formátu JSON.

Výsledkem práce je tedy aplikace, která přijme soubor ve formátu GEDCOM, zpracuje ho, uloží si veškeré potřebné informace do databáze a zobrazí vytvořené záznamy s~navrženými matrikami, které jsou řazené podle priorit. Díky zpracování archivů je možné se přes tyto navržené matriky prokliknout na jejich detail a rovnou v~nich vyhledávat. Aplikace disponuje i možností jednoduchého vyhledávání skrz záznamy na základě tagu nebo jména osob/rodin. Aplikace uživateli také nabízí poměrně sofistikovaný poznámkový systém, ve kterém si uživatel může vést poznámky jak k~matrikám, tak jednotlivým osobám či rodinám a propojit tyto poznámky mezi sebou.

V~budoucnu by se aplikace mohla rozšířit o~několik nových funkcionalit. Jednou z~nich by byla možnost uživatele dopisovat nalezené informace přímo k~záznamům a připisovat je tak do databáze. Při tom by se na základě nových informací mohl provést přepočet a mohli by tak být navrženy k~určitým záznamům nové matriky. Dalším rozšířením, souvisejícím s~předchozím, by mohlo být, že by aplikace mohla vyrobit nový GEDCOM soubor, který by obsahoval nově vyplněné informace a uživatel by si jej mohl stáhnout pro příští využití. Aplikace by se také dala rozšířit o~aliasy území, tedy názvy týkající se nějakých území v~cizím jazyce. Existuje mnoho dalších způsobů, kterými by se aplikace mohla rozšířit.

Vypracování této práce mi přineslo mnoho nových zkušeností. Především část přípravy dat, kdy jsem se musel naučit pracovat s~daty poskytovanými státem. Při zpracování archivů jsem se naučil, jak extrahovat data z~webových stránek pomocí jazyka Python, což pro mě do této doby bylo neprobádané území. Zlepšila se mi schopnost pracovat se všemi použitými jazyky a také schopnost sám rozebrat a řešit problémy, které nastávali při implementaci. Přínosem pro mě také bylo rozšíření obzorů v~rámci genealogie.