\chapter{Obsah přiloženého paměťového média}
\begin{itemize}
	\item \emph{src/}
	\begin{itemize}
		\item Adresářová struktura Laravel aplikace.
	\end{itemize}
	\item \emph{src/python\_scripts/}
	\begin{itemize}
		\item Python skripty pro parser a matcher.
	\end{itemize}
	\item \emph{territories/}
	\begin{itemize}
		\item Skripty pro zpracování území.
	\end{itemize}
	\item \emph{scrapers/}
	\begin{itemize}
		\item Python skripty pro extrakci dat z~online archivů.
	\end{itemize}
	\item \emph{scrapers/Jsons}
	\begin{itemize}
		\item Zpracované JSON soubory jednotlivých archivů.
	\end{itemize}
	\item \emph{doc/}
	\begin{itemize}
		\item Text technické zprávy.
	\end{itemize}
\end{itemize}

\chapter{Lokální instalace}
Tato příloha obsahuje návod pro zprovoznění aplikace na lokálním systému. Způsob nasazení aplikace na server je tomu lokálnímu velice podobný, jen je potřeba nastavit pár věcí navíc.

\subsubsection{Prerekvizity}
\begin{itemize}
	\item \textbf{Python} v3.10 + pip
	\item \textbf{PHP} v8.1
	\item \textbf{Composer}
	\item \textbf{MySQL}
	\item \textbf{Bash}
\end{itemize}

\subsubsection{Nastavení konfiguračního souboru}
Je zapotřebí nastavit konfigurační soubor Laravel aplikace s~názvem .env. Na Linuxových OS může být tento soubor skrytý před zobrazením. V~rámci tohoto souboru je zapotřebí nastavit v~sekci týkající se databáze parametry \verb|DB_USERNAME| a \verb|DB_PASSWORD|. Tedy uživatelské jméno a heslo k~účtu na MySQL serveru.

K~nachystání potřebných částí slouží shell skript \verb|setup.sh|. V~tomto skriptu je potřeba vyplnit databázové uživatelské jméno a heslo. Skript se poté spouští v~kořenovém adresáři příkazem:
\begin{itemize}
	\item \verb|./setup.sh|
\end{itemize}

V~následujících sekcích je vysvětlen postup tohoto skriptu.

\subsubsection{Instalace modulů pro jazyk Python}
Skript nejdříve provede nainstalování všech potřebných modulů pro jazyk Python za pomocí pip instalátoru.

\begin{itemize}
	\item \verb|pip install "nazev\_modulu"|
\end{itemize}

\subsubsection{Databáze}
Dále \verb|setup.sh| spouští příkaz:

\begin{itemize}
	\item \verb|mysql -uUSERNAME -pPASSWORD < SQL.sql|
\end{itemize}

Tento příkaz spustí na databázovém serveru queries ze souboru SQL.sql. Tyto queries vytvoří databázové schéma s~názvem \verb|xpopdo00|. Do tohoto schématu vytvoří tabulky a vloží do tabulek počáteční data.

\subsubsection{Instalace závislostí}
Zaváděcí skript dál přejde do složky \verb|src|, kde spouští příkaz: 

\begin{itemize}
	\item \verb|composer install|
\end{itemize}

který nainstaluje závislosti nutné ke správnému fungování frameworku Laravel.

\subsubsection{Spuštění aplikace}
Jako poslední skript provede celkem 3 příkazy:
\begin{itemize}
	\item \verb|php artisan route:clear|
	\item \verb|php artisan route:cache|
	\item \verb|php artisan serve|
\end{itemize}

První dva příkazy vyčistí a vytvoří nové routy. Poslední příkaz spustí aplikaci, která poběží na lokální instanci na adrese:

\begin{itemize}
	\item \verb|http://127.0.0.1:8000|
\end{itemize}

